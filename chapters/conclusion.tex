From the three different architectures tested in my thesis, ResNet came out as clear winner, being able to make predictions almost as accurate as the baseline network. This is a great starting point for future attempts to optimize the deep models further and clearly shows that it is possible to use more complex neural networks for X-ray astronomy. The pipeline I built for training and testing a deep neural network was made in such a way that it is easy to rerun the training with different parameters such as different learning rates, custom weights, validation loss monitoring to spot overfitting earlier during training and many more. This allows to further investigate which parameters can help to improve accuracy to beat the baseline CNN. It is even possible to try different neural network models apart from VGG, ResNet and EfficientNet to see if there might be other architectures that work for galaxy cluster mass estimation. For future attempts at improving these deep models, I would suggest to start with ResNet50V2 instead of the more accurate ResNet152V2 because it was almost as accurate as the complex model but with way less training time. Because of the similarities of the two models, I expect the same optimizations to work on both models which makes it unnecessarily more complicated to try to find them for the bigger model first. It would be very interesting to see if a neural network powered hyperparameter tuning would help to find better results in the future. \\
One thing that I find especially interesting is if there is a big difference in accuracy with no pretraining of the deep neural networks. I can imagine that the pretraining might limit the possible outcomes of the training and thus training without it could improve accuracy. Unfortunately, the time frame for a Bachelor's Thesis was not enough to test all of this.\\
Another very interesting investigation would be to see whether the model excels at certain galaxy cluster features such as big or small masses or specific redshifts. One could also use a way bigger dataset for training because it is possible to simulate any number of galaxy clusters and see the outcomes of this. Also changing the resolution of the input images or the frequency bands used could have interesting outcomes. \\ 

Astronomy in particular is driven by enormous amounts of data that is simply not possible to be scanned through by humans and machine learning is able to not only detect features on a human level of accuracy, but way better. Because most of astronomy's data is in the form of images, CNNs will play a key role in future discoveries because they proved to be exceptionally good at extracting features. With the rate of improvement seen over the last years, even better models and faster GPUs will accelerate the trend of using CNNs in astronomy and science in general.  \\

After over 40 days of total training time I have gained a lot of insight into the training of neural networks and especially the problems that come with it. Although not reflected in this paper, most of this thesis work was put into figuring out how to solve never ending programming problems in order to get the models to train. But once the training pipeline was built, it was possible to train multiple models at a time. I hope this pipeline will help future development and optimization of deep neural networks in order to improve galaxy mass estimation. 