Not long ago, I remember that there were websites from science and astronomy institutes such as \textit{Galaxy Zoo}\footnote{https://www.zooniverse.org/projects/zookeeper/galaxy-zoo/} where you were able to help them spot certain features from observational data. For instance, you were given a brief introduction into the different shapes of galaxies and then had to scan through a never ending stream of galaxy images and spot galaxies of a certain type. This was necessary because astronomy lives and breathes with enormous amounts of data. Way more images have been taken from space-observatories than a single human could ever scan through in a life-time. Although there have been human-developed algorithms to scan trough data for quite some time now, they are solely developed for one purpose only. For example an algorithm could be set up that is able to classify a supernova based on its luminosity over a certain time-span. It is not possible, at least not with some effort, to use this algorithm for something completely different. With AI, you can use architectures for all kinds of problems and are even able to get higher accuracies than the algorithms. With the development of ever more accurate neural networks for image classification and their success in recognition competitions, all the data collected so far can be used to train algorithms to spot features that we did not even know existed. While a human might be able to scan through this data for a few hours, trained neural networks can analyse thousands of images per second for as long as they are given energy. Knowing this, it only makes sense that machine learning is getting more popular amongst astrophysicists day by day and newly developed architectures find their way into science. This however is not a quick and easy process. Neural networks can be very challenging to train because all you can do is to try to lead them into the right direction of learning useful features and connections. If the algorithm is not able to find anything, it is not possible to tell us why exactly. Because of this, training a neural network is a tedious exercise that requires a lot of failed training runs and optimization to work. Neural network architectures can be very different from one another. Simple models are quicker in training but can lack complexity for more difficult tasks while models with more layers which in theory can be more accurate, are way harder to train and optimize. \\

Using a relatively simple neural network, \citet{Krippendorf_2023} were able to use the X-ray data of galaxy clusters from simulations for the eROSITA space mission to obtain the masses of these clusters. If successful, more complex neural networks could improve these estimation even further. In my thesis I want to study three different very successful deep neural network architectures to see whether they are able to detect galaxy cluster features reliably and estimate the clusters' masses accurately. For this, I will explain the basics of galaxy cluster X-ray emission and machine learning for image recognition first. Then I will give an impression of what it takes to train a neural network and which parameters have to be chosen for it to train well. At the end I will compare their results with the simple neural network used by the scientists and provide some ideas on what to do next with these models.