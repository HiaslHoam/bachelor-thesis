Within the last few years, the usage of convolutional neural networks (CNNs) has exploded. While in 2018 around 50 papers where published who used CNNs for astrophysical problems, the number of publications has tripled to around 150 in 2022\footnote{Numbers from https://arxiv.org/}. To this date, there are only a few approaches to infer galaxy masses with CNNs. Usually, a simulation set is needed in order to train the neural network. Because by training with data from let's say the mass estimation mentioned in \cref{mass_est}, the neural network would never be able to get a more accurate result than the data it was being trained with. That's why simulations are needed where the exact cluster mass is known. Luckily, these are being generated anyway for many missions such as Chandra and eROSITA to be able to prepare data analysis methods before the start of the actual mission. For instance \citet{Ntampaka_2018} used mock data for the Chandra X-ray satellite to train a CNN. Especially important for my thesis is the development of a neural network for eROSITA's \textit{Final Equatorial-Depth Survey} (eFEDS). \citet{Krippendorf_2023} used galaxy cluster data specifically simulated for eFEDS' observational data to train a CNN for mass estimation. The simulation data includes ten frequency bands and redshift data (see \cref{data_catalogue}). This approach used a comparably simple architecture, being the baseline for my development of a deep neural network using the same data set (see \cref{training}). Being able to infer galaxy cluster masses in many different ways is a great way to improve the accuracy of missions' data. AI and especially CNNs play an important role in receiving higher accuracies. How they work and why they are used is being discussed in \cref{CNN Structure}.